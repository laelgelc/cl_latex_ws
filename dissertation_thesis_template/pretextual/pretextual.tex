\selectlanguage{brazilian} % Comment this line if the document is expected to be in English

\begin{titlepage}
    \begin{center}
        %{\large \MakeUppercase{Pontifical Catholic University of São Paulo}}\\
        {\large \MakeUppercase{Pontifícia Universidade Católica de São Paulo}}\\
        {\large \MakeUppercase{PUC-SP}}\\
        \vspace{5cm}
        %{\large \MakeUppercase{Name Surname}}\\
        {\large \MakeUppercase{Nome Sobrenome}}\\
        \vspace{5cm}
        %{\Large \MakeUppercase{Title: title}}\\
        {\Large \MakeUppercase{Título: título}}\\
        \vspace{5cm}
        %{\large \MakeUppercase{Master of Arts in Applied Linguistics and Language Studies}}
        {\large \MakeUppercase{Mestrado em Linguística Aplicada e Estudos da Linguagem}}
        \vfill
        {\large \MakeUppercase{São Paulo}}
        
        {\large 2025}
    \end{center}
\end{titlepage}

\begin{titlepage}
    \begin{center}
        %{\large \MakeUppercase{Pontifical Catholic University of São Paulo}}\\
        {\large \MakeUppercase{Pontifícia Universidade Católica de São Paulo}}\\
        {\large \MakeUppercase{PUC-SP}}\\
        \vspace{5cm}
        %{\large \MakeUppercase{Name Surname}}\\
        {\large \MakeUppercase{Nome Sobrenome}}\\
        \vspace{5cm}
        %{\Large \MakeUppercase{Title: title}}\\
        {\Large \MakeUppercase{Título: título}}\\
        \vspace{3cm}
        \begin{adjustwidth}{0.5\textwidth}{0cm}
            %Dissertation submitted to the dissertation committee of the Pontifical Catholic University of São Paulo in partial fulfillment of the requirements for the degree of Master of Arts in Applied Linguistics and Language Studies, under the supervision of Professor Tony Berber Sardinha, Ph.D.
            Dissertação apresentada à banca examinadora da Pontifícia Universidade Católica de São Paulo como exigência parcial para obtenção do título de Mestre em Linguística Aplicada e Estudos da Linguagem, sob orientação do Prof. Dr. Antonio Paulo Berber Sardinha.
        \end{adjustwidth}
        \vfill
        {\large \MakeUppercase{São Paulo}}
        
        {\large 2025}
    \end{center}
\end{titlepage}

\begin{titlepage}
    %I authorise, exclusively for academic and scientific purposes, the total or partial reproduction of this master's dissertation by photocopying or electronic processes.
    Autorizo, exclusivamente para fins acadêmicos e científicos, a reprodução total ou parcial desta dissertação de mestrado por processos fotocopiadores ou eletrônicos.

    \vspace{2cm}
    %Signature: \makebox[0.75\textwidth]{\hrulefill}\\
    Assinatura: \makebox[0.75\textwidth]{\hrulefill}\\
    %Date: 10/09/2025\\
    Data: 10/09/2025\\
    %E-mail: example@example.com\\
    E-mail: exemplo@exemplo.com\\
    %Lattes Curriculum: \href{http://lattes.cnpq.br/9204618373258699}{http://lattes.cnpq.br/9204618373258699}\\
    Currículo Lattes: \href{http://lattes.cnpq.br/9204618373258699}{http://lattes.cnpq.br/9204618373258699}\\
    \vfill
    \begin{center}
        % Gerenciador de ficha catalográfica
        % http://biblio2.pucsp.br/ficha/?_ga=2.154384056.1415767632.1628681585-1429258994.1628681585

        %\begin{tabular}{|p{0.85\linewidth}|}
        %    \hline
        %    %[Bibliographic information -- to be included after dissertation defence]\\
        %    [Informação bibliográfica -- a ser incluída após a defesa da dissertação]\\
        %    \\ \hline
        %\end{tabular}

        \begin{tabular}{|p{1cm} p{11.5cm}|}
            \hline
             & \\
             %& \texttt{\small Surname, Name} \\
             & \texttt{\small Sobrenome, Nome} \\
            %\texttt{\small Y19f} & \texttt{\small Title: title. / \texttt{\small Name Surname. --- São Paulo: [s.n.], 2025.}} \\
            \texttt{\small Y19f} & \texttt{\small Título: título. / \texttt{\small Nome Sobrenome. --- São Paulo: [s.n.], 2025.}} \\
             & \texttt{\small 72p. il. ; 29,7 x 21,0 x 0,73 cm.} \\
             & \\
             %& \texttt{\small Supervisor: Professor Antonio Paulo Berber Sardinha, Ph.D..} \\
             & \texttt{\small Orientador: Professor Doutor Antonio Paulo Berber Sardinha.} \\
             %& \texttt{\small Dissertation (Master's) --- Pontifical Catholic University of São Paulo, Applied Linguistics and Language Studies Graduate Programme.} \\
             & \texttt{\small Dissertação (Mestrado) --- Pontifícia Universidade Católica de São Paulo, Programa de Estudos Pós-Graduados em Linguística Aplicada e Estudos da Linguagem.} \\
             & \\
             %& \texttt{\small \small 1. keyword 1. 2. key word 2. 3. keyword 3. 4. keyword 4. I. Berber Sardinha, Ph.D., Professor Antonio Paulo. II. Pontifical Catholic University of São Paulo, Applied Linguistics and Language Studies Graduate Programme. III. Title.} \\
             & \texttt{\small \small 1. palavra-chave 1. 2. palavra-chave 2. 3. palavra-chave 3. 4. palavra-chave 4. I. Berber Sardinha, Professor Doutor Antonio Paulo. II. Pontifícia Universidade Católica de São Paulo, Programa de Estudos Pós-Graduados em Linguística Aplicada e Estudos da Linguagem. III. Título.} \\
             & \\
             & \hspace{8cm} \texttt{\small CDD 410} \\
             & \\
             \hline
        \end{tabular}
    \end{center}
\end{titlepage}

\begin{titlepage}
    Nome Sobrenome\\
    %Title: title\\
    Título: título\\
    %Approved: 10/09/2025
    Aprovada em: 10/09/2025

    \vspace{2cm}
    %Dissertation submitted to the dissertation committee of the Pontifical Catholic University of São Paulo in partial fulfillment of the requirements for the degree of Master of Arts in Applied Linguistics and Language Studies, under the supervision of Professor Tony Berber Sardinha, Ph.D.
    Dissertação apresentada à banca examinadora da Pontifícia Universidade Católica de São Paulo como exigência parcial para obtenção do título de Mestre em Linguística Aplicada e Estudos da Linguagem, sob orientação do Prof. Dr. Antonio Paulo Berber Sardinha.

    \vspace{4cm}
    %Dissertation Committee
    Banca Examinadora

    \vspace{3cm}
    \makebox[0.75\textwidth]{\hrulefill}\\
    %Prof. Name Surname, Ph.D.
    Prof\textsuperscript{a}. Dr\textsuperscript{a}. Nome Sobrenome

    \vspace{3cm}
    \makebox[0.75\textwidth]{\hrulefill}\\
    %Prof. Name Surname, Ph.D.
    Prof\textsuperscript{a}. Dr\textsuperscript{a}. Nome Sobrenome

\end{titlepage}

\begin{titlepage}
    \begin{center}
        \vspace*{11cm}
        %{To Name}
        {A Nome}
        \vfill
    \end{center}
\end{titlepage}

\begin{titlepage}
    \vspace*{10cm}
    \begin{adjustwidth}{0.5\textwidth}{0cm}
        %The present work was carried out with the support of the National Council for Scientific and Technological Development (CNPq), Grant \#131464/2023-0.
        O presente trabalho foi realizado com o apoio do Conselho Nacional de Desenvolvimento Científico e Tecnológico (CNPq), Processo 131464/2023-0.
    \end{adjustwidth}
\end{titlepage}

\begin{titlepage}
    \begin{center}
        %{\large \MakeUppercase{Acknowledgements}}
        {\large \MakeUppercase{Agradecimentos}}
    \end{center}
    \vspace{1cm}

    \lipsum[1-2]

\end{titlepage}

\begin{titlepage}
    \vspace*{\fill}
    %\epigraph{`[R]egisters' are cultural categories, not scientific categories. These categories can be described for their typical situational and linguistic characteristics. But they are not defined in those terms. In fact, registers do not have definitions in terms of their necessary characteristics. Rather, cultures and languages evolve naturally in terms of such categorical organizations, without any scientific basis.}{\textcite[p.~16]{biberWhatRegisterAccounting2023}}
    \epigraph{`[R]egistros' são categorias culturais, não categorias científicas. Essas categorias podem ser descritas por suas características situacionais e linguísticas típicas. Mas elas não são definidas nesses termos. Na verdade, registros não possuem definições em termos de características necessárias. Em vez disso, culturas e línguas evoluem naturalmente em termos de tais organizações categóricas, sem qualquer fundamento científico.}{\textcite[p.~16]{biberWhatRegisterAccounting2023}}
\end{titlepage}

%\selectlanguage{brazilian} % Uncomment this line if the document is expected to be in English

\title{\Large \MakeUppercase{Título: título}}
\author{Nome Sobrenome} % Your name here
\date{2025}
\maketitle
\renewcommand{\abstractname}{Resumo}
\begin{abstract}
    \lipsum[1]

    \vspace{1em}
    \textbf{Palavras-chave}: palavra-chave 1, palavra-chave 2, palavra-chave 3, palavra-chave 4
\end{abstract}

\selectlanguage{english}

\title{\Large \MakeUppercase{Title: title}}
\author{Name Surname} % Your name here
\date{2025}
\maketitle
\begin{abstract}
    \lipsum[1]

    \vspace{1em}
    \textbf{Keywords}: keyword 1, keyword 2, keyword 3, keyword 4
\end{abstract}

\selectlanguage{brazilian} % Comment this line if the document is expected to be in English

\listoffigures
\thispagestyle{empty}
\listoftables
\thispagestyle{empty}
\tableofcontents
\thispagestyle{empty}
\clearpage
\pagenumbering{arabic}
\setcounter{page}{15} % Set the initial page number
