\begin{longtable}[c]{ |p{8cm}|p{8cm}|p{8cm}| }
    \captionsetup{justification = raggedright, singlelinecheck = false}
    \caption{Análise de domínio e desenho de corpus}
    \label{tab:domain_analysis_and_corpus_design1_ptbr}
    \\ \hline
    \textbf{\makebox[8cm][c]{Descrição do Domínio}} & \textbf{\makebox[8cm][c]{Operacionalização do Domínio}} & \textbf{\makebox[8cm][c]{Planejamento da Amostra}} \\ \hline
    \endfirsthead
    \hline
    \textbf{\makebox[8cm][c]{Descrição do Domínio}} & \textbf{\makebox[8cm][c]{Operacionalização do Domínio}} & \textbf{\makebox[8cm][c]{Planejamento da Amostra}} \\ \hline
    \endhead
    \hline
    \multicolumn{3}{r}{Continua na próxima página} \\
    \endfoot
    \endlastfoot
    \textbf{Métodos}
    \begin{itemize}
        \item Experiência do pesquisador
        \item Pesquisas acadêmicas anteriores
        \item Buscas baseadas na Web/IA
    \end{itemize}
    \textit{Artigos de pesquisa}
    \begin{itemize}
        \item Taxonomias de disciplinas acadêmicas (Ex.: \href{https://www.nsf.gov/}{NSF}; \href{https://www.gov.br/cnpq/pt-br}{CNPq})
        \item Normas de publicação científica (Ex.: normas de periódicos acadêmicos)
    \end{itemize}
    \textit{Textos gerados por IA}
    \begin{itemize}
        \item Organizações de pesquisa em IA
    \end{itemize}
     &
    \textbf{Fronteiras}
    \begin{itemize}
        \item Disciplina acadêmica
        \item Inglês-como-Língua-Internacional: Periódicos de qualidade em inglês
        \item Inglês-como-Língua-Regional: \href{https://preprints.scielo.org/index.php/scielo}{Repositório SciELO Pré-prints}
        \item Esquema funcional de seções
        \item Autoria por IA
        \begin{itemize}
            \item Modelo e versão de LLM: ChatGPT 4.1
            \item Assistente de escrita
        \end{itemize}
        \item Data de publicação: Antes do advento do ChatGPT, em 30 de novembro de 2022
    \end{itemize}
     &
    \textbf{Unidade de amostragem}
    \begin{itemize}
        \item Um parágrafo, com no mínimo 10 palavras
    \end{itemize}
     \\
    \textbf{Fronteiras}
    \begin{itemize}
        \item Registro acadêmico escrito
        \item Voltado para pesquisa
        \item Baseado em evidências
        \item Revisado por pares
    \end{itemize}
     &
    \textbf{Estratos}
    \begin{itemize}
        \item Taxonomia de disciplinas acadêmicas do \href{https://www.gov.br/cnpq/pt-br}{CNPq}: Ciências Exatas e da Terra; Ciências Biológicas; Engenharias; Ciências da Saúde; Ciências Agrárias; Ciências Sociais Aplicadas; Ciências Humanas; Linguística, Letras, e Artes
        \item Esquemas de seções
        \begin{itemize}
            \item Funcional (em língua inglesa)
            \begin{itemize}
                \item Abstract
                \item Introduction
                \item Literature Review
                \item Methodology
                \item Results
                \item Discussion
                \item Conclusion
                \item Acknowledgements
                \item Comments to the Editor
            \end{itemize}
            \item Temático: Expresso como uma combinação de termos do esquema funcional (em língua inglesa). Ex.:
            \begin{itemize}
                \item Abstract
                \item Introduction, Literature Review
                \item Methodology
                \item Results
                \item Discussion, Conclusion
                \item Acknowledgements
            \end{itemize}
        \end{itemize}
        \item Autoria
        \begin{itemize}
            \item Produzido por humanos
            \item Revisado por IA
        \end{itemize}
    \end{itemize}

     &
    \textbf{Métodos de amostragem}
    \begin{itemize}
        \item Amostra balanceada entre
        \begin{itemize}
            \item Disciplinas acadêmicas
            \item Seção
        \end{itemize}
        \item Procedimento para o estrato de disciplina acadêmica
        \begin{enumerate}
            \item Classifique as disciplinas pelo número de parágrafos
            \item Balanceie o corpus de acordo com a disciplina que:
            \begin{itemize}
                \item contribui com menos parágrafos
                \item maximiza o tamanho do corpus
                \item causa a menor redução no número de disciplinas
            \end{itemize}
            \item Exclua as disciplinas que estão abaixo da disciplina do passo anterior
        \end{enumerate}
        \item Procedimento para o estrato de seção
        \begin{enumerate}
            \item Identifique os padrões de seções que são comuns a todas as disciplinas
            \item Exclua os padrões divergentes
        \end{enumerate}
    \end{itemize}
     \\
    \textbf{Categorias}
    \begin{itemize}
        \item Modo: Escrito
        \item Variedade da língua inglesa
        \begin{itemize}
            \item Inglês-como-Língua-Internacional
            \item Inglês-como-Língua-Regional
        \end{itemize}
        \item Estágio de revisão
        \begin{itemize}
            \item Pré-print
            \item Publicado
        \end{itemize}
        \item Autoria
        \begin{itemize}
            \item Produzido por humanos
            \item Sintético (Gerado por IA)
            \begin{itemize}
                \item Modelo e versão de LLM de IA
                \item Natureza da aplicação de IA: Assistente de escrita
            \end{itemize}
        \end{itemize}
        \item Normas de publicação científica
        \begin{itemize}
            \item Particular a cada disciplina acadêmica
            \item Particular a cada veículo de publicação
        \end{itemize}
        \item Organização
        \begin{itemize}
            \item Esquema funcional (canônico) de seções
            \item Esquema temático de seções
        \end{itemize}
    \end{itemize}
     &
    
     &
    
     \\ \hline
    \caption*{Fonte: \textcite{egbertDomainConsiderations2022}}
\end{longtable}

\begin{longtable}[c]{ |p{8cm}|p{8cm}| }
    \captionsetup{justification = raggedright, singlelinecheck = false}
    \caption{Avaliação}
    \label{tab:domain_analysis_and_corpus_design2_ptbr}
    \\ \hline
    \textbf{\makebox[8cm][c]{Domínio $\leftarrow$ Domínio Operacional}} & \textbf{\makebox[8cm][c]{Domínio Operacional $\leftarrow$ Corpus}} \\ \hline
    \endfirsthead
    \hline
    \textbf{\makebox[8cm][c]{Domínio $\leftarrow$ Domínio Operacional}} & \textbf{\makebox[8cm][c]{Domínio Operacional $\leftarrow$ Corpus}} \\ \hline
    \endhead
    \hline
    \multicolumn{2}{r}{Continua na próxima página} \\
    \endfoot
    \endlastfoot
    \begin{itemize}
        \item Taxonomias de disciplinas acadêmicas diferentes da do \href{https://www.gov.br/cnpq/pt-br}{CNPq} não estão representadas
        \item Artigos organizados de acordo com o esquema temático de seções podem não ser totalmente representados pela combinação de termos do esquema funcional
        \item Modelos de LLM diferentes do ChatGPT 4.1 não são considerados
        \item Artigos em Inglês-como-Língua-Regional que não sejam provenientes do \href{https://preprints.scielo.org/index.php/scielo}{repositório de pré-prints da SciELO} não são considerados
    \end{itemize}
     &
    \begin{itemize}
        \item Parágrafos com menos de 10 palavras não são considerados
        \item Títulos, subtítulos, cabeçalhos e rodapés não são considerados
        \item Elementos como tabelas, figuras, fórmulas, referências e apêndices não são considerados
        \item As disciplinas de Engenharias, Ciências Exatas e da Terra, e Ciências Agrárias estão excluídas
        \item Artigos com padrões de seções não comuns a todas as disciplinas não são considerados
    \end{itemize}
     \\ \hline
    \caption*{Fonte: \textcite{egbertDomainConsiderations2022}}
\end{longtable}
