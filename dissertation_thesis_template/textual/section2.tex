\section{Revisão de Literatura}

\textcite{berbersardinhaAIgeneratedHumanauthoredTexts2024} \lipsum[1]

\subsection{Subseção 1}

\lipsum[1]

\subsubsection{Subsubseção 1}

\lipsum[1]

\begin{itemize}
    \item
    \item
\end{itemize}

\lipsum[1]

\begin{enumerate}
    \item
    \item
\end{enumerate}

\lipsum[1]

A Dimensão 1, Produção Interativa versus Produção Informacional, distingue entre textos que focam no envolvimento pessoal, interação e afeto (por exemplo, conversas, cartas pessoais) e aqueles que apresentam informações de maneira mais distante e informacional (por exemplo, prosa acadêmica, relatórios). Essa dimensão é ilustrada graficamente na Figura \ref{fig:merged_dim1_biber_ptbr}: a parte inferior exibe os traços linguísticos com cargas salientes na dimensão, enquanto a parte superior representa uma escala contínua de variação, com registros de escore alto no lado esquerdo e registros de escore baixo no lado direito. Registros que se destacam na Dimensão 1 fazem uso estatisticamente significativo de traços como \textit{private verbs}, \textit{contractions} e \textit{present tense verbs} para refletir interação interpessoal, foco na postura pessoal e circunstâncias de produção em tempo real. Por outro lado, registros que estão no extremo direito da escala são caracterizados pelo uso significativo de \textit{nouns}, palavras mais longas, \textit{attributive adjectives} e riqueza vocabular (\textit{type-token ratio}), refletindo um foco informacional, a integração cuidadosa das informações no texto e uma escolha lexical precisa.

\import{diagrams/}{merged_dim1_biber_ptbr}

\begin{quote}
    \small Os valores de F e p representam os resultados de uma ANOVA, que testa se há diferenças estatisticamente significativas entre os registros em relação às médias dos escores das dimensões. O valor de R\textsuperscript{2} é uma medida direta de força ou importância. Esse valor indica a porcentagem da variância entre os escores das dimensões que pode ser explicada pelo conhecimento das categorias de registro. \parencite[p.~834, tradução nossa]{biberMultidimensionalApproaches2009}
\end{quote}

\lipsum[1]

As Tabelas \ref{tab:framework_for_domain_analysis_and_corpus_design1_ptbr} e \ref{tab:framework_for_domain_analysis_and_corpus_design2_ptbr} resumem este arcabouço para análise de domínio e desenho de corpus.

\begin{landscape}
    %{\footnotesize \import{tables/}{domain_analysis_and_corpus_design_template}}
    {\footnotesize \import{tables/}{domain_analysis_and_corpus_design_template_ptbr}}
\end{landscape}

\lipsum[1]

%\import{maths/}{sample_size_formula}
\import{maths/}{sample_size_formula_ptbr}

\lipsum[1]
